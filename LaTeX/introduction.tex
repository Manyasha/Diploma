\chapter*{Вступ}
\addcontentsline{toc}{chapter}{Вступ}

Повштухом до росту зацікавленості теорією та практикою математичного програмування стало відкриття в 1947 році обчислювального методу розв'язування завдань лінійного програмування. Цей чисельний метод був названий симплекс методом.

Одночасно з підвищенням інтересу до лінійного програмування, популярність здобували й нелінійні задачі. У 1951 році була опублікована праця Куна і Таккера, у якій було викладено необхідні та достатні умови оптимальності розв'язку нелінійних задач. Саме ця праця стала фундаментом до багатьох наступних робіт нелінійного програмування. 

Починаючи з 1954 року почала з'являтися велика кількість праць, які присвячені квадратичному програмуванню. Але у більшості робіт були майже однакові обчислювальні алгоритми.

На сучасному етапі розвитку науки та комп'ютерних технологій теорія чисельних методів нелінійного програмування є досить розвиненою. Проте ще досить важко дати чіткі рекомендації щодо застосування того чи іншого методу. Тому теорія методів оптимізації продовжує розвиватися, з'являються нові методи, які мають свої переваги у порівнянні з попередніми.

У літературі описані багатокрокові методи, а саме - двокроковий метод спряжених градієнтів. Цей метод вважається досить ефективним для задач великої розмірності. Метод спряжених градієнтів має перевагу перед однокроковими градієнтрими методами, бо він у більшій мірі враховує геометричні властивості цільової функції. Опираючись на цю інформацію, можна піти далі і розглянути трикрокові, чотирикрокові, п'ятикрокові і т.д. методи.

У роботі розглянуто чотирикроковий метод, дано чисельну порівняльну характеристику у порівнянні з трикроковим методом.

\newpage
\textbf{Мета роботи} – дослідити чотирикроковий метод мінімізації функцій.

Відповідно до поставленої мети в роботі вирішуються такі конкретні \textbf{завдання}:
\begin{enumerate}
	\item розглянути теоретичні засади чотирьохкрокового методу;
	\item перевірити його доцільність на прикладах.
\end{enumerate}  

\textbf{Об'єкт дослідження} – пошук мінімуму функцій.

\textbf{Предмет} – чотирикроковий метод мінімізації функцій.
