\chapter{Чотирьохкроковий метод мінімізації функцій багатьох змінних без обмежень}

Для розв'язання задачі мінімізації функції багатьох змінних без обмежень (1.3.1) розглянемо чотирьохкроковий метод з таким алгоритмом:
$$ x^{k + 1} = x^k + \beta_k s^k, \; k = 0, 1,\ldots $$
$$
S = \quad
\begin{cases}
-\phi' (x^k) & , k = 0 \\
-\phi' (x^k) + \gamma_1^{k-1}s^{k-1} & , k = 1 \\
-\phi' (x^k) + \gamma_1^{k-1}s^{k-1} + \gamma_2^{k-2}s^{k-2} & , k = 2 \\
-\phi' (x^k) + \gamma_1^{k-1}s^{k-1} + \gamma_2^{k-2}s^{k-2} +  \gamma_3^{k-3}s^{k-3} & , k = 3, \ldots
\end{cases}	\eqno(2.0.1)
$$
де $ x^0, x^1, \dotsc , x^k, \dotsc $ - послідовні наближення \\
$ s^0, s^1, \dotsc , s^k, \dotsc $ - напрямки спуску \\
$ \beta_k, \gamma_j^m (j = \overline{1,3}) $ - числові параметри 

Параметр $\beta_k$ будемо визначати з умови (1.3.3).
\\
\begin{defn}\label{conjugatev}
	Вектори $ s' \text{ і} \; s'' $ називаються спряженими (відносно матриці $A$), якщо вони відмінні від нуля і $(As', s'') = 0$. \\
	Вектори $ s^0, s^1, ..., s^m $ називаються взаємно спряженими (відносно матриці $A$), якщо всі вони відмінні від нуля і $(As', s'') = 0, i \neq j, 0 \leq i,j \geq m$. Матриця $A$ вважається симетричною і додатньо визначеною $(A > 0).$
\end{defn}

\section{Теоретичні положення методу}

Розглянемо деякі властивості методу при умові, що функція $\phi(x)$ є квадратичною.
$$  \phi(x) = \frac{1}{2} (Ax,x) + (b,x) + c \eqno(2.1.1) $$
Побудуємо систему взаємно спряжених напрямків за правилом (2.0.1). 
$$ 0 = (s^k, As^{k-1}) = -(\phi'(x^k), As^{k-1}) + \gamma_1^{k-1}(s^{k-1}, As^{k-1}) + $$
$$ + \gamma_2^{k-2}\underbrace{(s^{k-2}, As^{k-1})}_{0} + \gamma_3^{k-3}\underbrace{(s^{k-3}, As^{k-1})}_{0}  \Longrightarrow$$ 
$$\gamma_1^{k-1} = \dfrac{(\phi'(x^k), As^{k-1})}{(s^{k-1}, As^{k-1})} \eqno(2.1.2)$$
Внаслідок того, що матриця $A$ додатньо визначена, знаменник у (2.1.2) не дорівнює нулеві. 
Аналогічно отримаємо: 
$$ 0 = (s^k, As^{k-2}) = -(\phi'(x^k), As^{k-2}) + \gamma_1^{k-1}\underbrace{(s^{k-1}, As^{k-2})}_{0} + $$
$$ + \gamma_2^{k-2}(s^{k-2}, As^{k-2}) + \gamma_3^{k-3}\underbrace{(s^{k-3}, As^{k-2})}_{0}  \Longrightarrow$$  
$$\gamma_2^{k-2} = \dfrac{(\phi'(x^k), As^{k-2})}{(s^{k-2}, As^{k-2})} \eqno(2.1.3)$$

$$ 0 = (s^k, As^{k-3}) = -(\phi'(x^k), As^{k-3}) + \gamma_1^{k-1}\underbrace{(s^{k-1}, As^{k-3})}_{0} + $$
$$ + \gamma_2^{k-2}\underbrace{(s^{k-2}, As^{k-3})}_{0} + \gamma_3^{k-3}(s^{k-3}, As^{k-3})  \Longrightarrow$$  
$$\gamma_3^{k-3} = \dfrac{(\phi'(x^k), As^{k-3})}{(s^{k-3}, As^{k-3})} \eqno(2.1.4)$$

\begin{thm} \label{t1}
	Для диференційовної функції $\phi(x)$ послідовність $\{x^k\}$, що побудована за (2.0.1), (2.1.2), (2.1.3), (2.1.4), така, що $$ (\phi'(x^{k+1}), s^k) = 0, \, k = 0, 1, \ldots \eqno(2.1.5)$$ 
\end{thm}
\begin{proof}[Доведення:]
	Враховуючи (2.0.1) отримаємо $Ax^{k+1} = Ax^k + \beta_kAs^k$. Так як $\phi'(x) = Ax + b$, то маємо $$ \phi'(x^{k+1}) = \phi'(x) + \beta_k  A  s^k \eqno(2.1.6)$$
	З (2.0.3) отримаємо, що 	
	\begin{align*}
	\dfrac{d}{d \beta} \phi(x^k + \beta s^k)&\Bigr|_{\beta = \beta_k} = 0 & , \; \beta_k& > 0 \\
	\dfrac{d}{d \beta} \phi(x^k + \beta s^k)&\Bigr|_{\beta < 0} \; \; \geq 0 & , \; \beta_k& = 0
	\end{align*}	
	Якщо $\beta_k > 0$, то
	$$ 0 = \dfrac{d}{d \beta} \phi(x^k + \beta s^k) \Bigr|_{\beta = \beta_k} = ( \phi'(x^k + \beta_k s^k), s^k) = (\phi'(x^{k+1}), s^k) $$
	Отже, отримали, що $(\phi'(x^{k+1}), s^k)  = 0, \; k = 0, 1, \ldots$
	
	Застосуємо метод індукції для доведення того, що співвідношення (2.1.5) справедливе і при $\beta_k = 0$:
	
	\begin{enumerate}
		{% 
			\renewcommand{\baselinestretch}{1.7}
			\selectfont
			\item $ 0 \leq \dfrac{d}{d \beta} \phi(x^0 + \beta s^0) \Bigr|_{\beta = 0} = ( \phi'(x^1), s^0) = (\phi'(x^0), -\phi'(x^0)) = $
			
			$ =  -\lVert \phi'(x^0) \rVert^2 \Rightarrow (\phi'(x^1), s^0)  = 0$
			\item припустимо, що $(\phi'(x^k), s^{k-1})  = 0$
			\item доведемо, що $(\phi'(x^{k+1}), s^k)  = 0, \beta_k = 0$
			
		}%

		Так, як $x^{k+1} = x^k$, то враховуючи (2.1.6) маємо:
		
		$$ 0 \leq \dfrac{d}{d \beta} \phi(x^k + \beta s^k) \Bigr|_{\beta = 0} =  ( \phi'(x^{k+1}), s^k) = ( \phi'(x^k), s^k) = $$
		$$ = ( \phi'(x^k), - \phi'(x^k) + \gamma_1^{k-1}s^{k-1} + \gamma_2^{k-2}s^{k-2} + \gamma_3^{k-3}s^{k-3}) = $$
		$$ = -( \phi'(x^k), \phi'(x^k)) + \gamma_1^{k-1}( \phi'(x^k),s^{k-1}) + $$
		$$ + \gamma_2^{k-2}( \phi'(x^k),s^{k-2}) + \gamma_3^{k-3}( \phi'(x^k),s^{k-3}) $$
		Враховуючи (2.1.6) і припущення індукції маємо:		
		$$( \phi'(x^k),s^{k-2}) = (\phi'(x^{k-1}) + \beta_{k-1}As^{k-1}, s^{k-2}) = (\phi'(x^{k-1}), s^{k-2}) + $$
		$$ + \beta_{k-1}(As^{k-1}, s^{k-2}) = (\phi'(x^{k-1}), s^{k-2}) = 0 $$
		$$( \phi'(x^k),s^{k-3}) = (\phi'(x^{k-1}) + \beta_{k-1}As^{k-1}, s^{k-3}) = (\phi'(x^{k-1}), s^{k-3}) + $$
		$$ + \beta_{k-1}(As^{k-1}, s^{k-3}) = (\phi'(x^{k-1}), s^{k-3}) = (\phi'(x^{k-2}) + \beta_{k-2}As^{k-2}, s^{k-3})  = $$
		$$ = (\phi'(x^{k-2}), s^{k-3}) + \beta_{k-2}(As^{k-2}, s^{k-3}) = (\phi'(x^{k-2}), s^{k-3}) = 0 $$		
		Отже, отримали:
		$$ 0 \leq (\phi'(x^{k+1}), s^k) = -\lVert \phi'(x^k) \rVert^2 \leq 0 $$
		Таким чином довели, що $ (\phi'(x^{k+1}), s^k) = 0$
	\end{enumerate}		
\end{proof}

\begin{thm} \label{t2}
	Вектори $\phi'(x^k) \; \text{i} \; \phi'(x^{k+1}) $ ортогональні, $ k = 0, 1, \ldots $ 
\end{thm}
\begin{proof}[Доведення:]
	Відомо, що квадратична функція (2.1.1) досягає мінімального значення при
	$$ \beta_k = -\dfrac{(\phi'(x^k), s^k)}{(As^k, s^k)} \eqno(2.1.7) $$
	\\
	Тоді враховуючи (2.1.6) та (2.1.7), отримаємо:	
	$$
	(\phi'(x^{k+1}), \phi'(x^k)) = (\phi'(x^k) + \beta_kAs^k,  \phi'(x^k)) = 
	$$
	$$ 
	 = (\phi'(x^k) - \dfrac{(\phi'(x^k), s^k)}{(As^k, s^k)}As^k,  \phi'(x^k)) = 
	$$
	$$	
	= (\phi'(x^k), \phi'(x^k)) - \dfrac{(\phi'(x^k), s^k)}{(As^k, s^k)}(As^k,  \phi'(x^k))
	$$		
	
	Розглянємо $ (\phi'(x^k), s^k) $.
	
	$$ (\phi'(x^k), s^k) = - (\phi'(x^k), \phi'(x^k)) + \gamma_1^{k-1}\underbrace{(\phi'(x^k), s^{k-1})}_{0} + $$
	$$ + \gamma_2^{k-2}\underbrace{(\phi'(x^k), s^{k-2})}_{0} + \gamma_3^{k-3}\underbrace{(\phi'(x^k), s^{k-3})}_{0} = 
	$$
	$$ = - (\phi'(x^k), \phi'(x^k)) \eqno(2.1.8) $$
	
	Далі, $$ (As^k, s^k) = (As^k, -\phi'(x^k) + \gamma_1^{k-1}s^{k-1} + \gamma_2^{k-2}s^{k-2} + \gamma_3^{k-3}s^{k-3} ) = $$
	$$ = - (As^k, \phi'(x^k)) + \gamma_1^{k-1}(As^k, s^{k-1}) + \gamma_2^{k-2}(As^k, s^{k-2}) + \gamma_3^{k-3}(As^k, s^{k-3}) = $$
	$$ =  - (As^k, \phi'(x^k)) \eqno(2.1.9) $$
	З урахуванням (2.1.8) і (2.1.9) отримаємо:	
	$$  (\phi'(x^{k+1}), \phi'(x^k)) =  (\phi'(x^k), \phi'(x^k)) - \dfrac{-(\phi'(x^k),\phi'(x^k))}{-(As^k,\phi'(x^k))}(As^k, \phi'(x^k)) = 0$$ 
\end{proof}

\begin{thm} \label{t3}
	$ \text{Нехай } x^0 \in \mathbb{R}^n , \text{ точки } x^1, x^2, \ldots , x^{n-1} \text{ і вектори } s^0, s^1, \ldots, s^{n-1} $ отримані за формулами  (2.0.1), (2.1.2), (2.1.3), (2.1.4) і 
	$ \phi'(x^k) \neq 0 \\ (k = \overline{0,n-1}) $, тоді вектори $  s^0, s^1, \ldots, s^{n-1} $ взаємно спряжені, а градієнти \\ $  \phi'(x^0), \phi'(x^1), \ldots,  \phi'(x^{n-1}) $ взаємно ортогональні.
\end{thm}
\begin{proof}[Доведення:]
	Теорему будемо доводити методом математичної індукції.
	\begin{enumerate}
		\item $  \phi'(x^0) \; і \; \phi'(x^1) $ - ортогональні внаслідок теореми \ref{t2} \\
		$s^0 \neq 0 $ - за умовою теореми \\
		$s^1 \neq 0 $, так як $s^1 = - \phi'(x^1) - \gamma_0\phi'(x^0) = 0 $, а це неможливо, враховуючи ортогональність  $  \phi'(x^0) \; і \; \phi'(x^1) $ \\
		спряженість $s^0 \; і \; s^1 $ отримаємо з (2.1.2), (2.1.3), (2.1.4) 
		\item Припустимо, що $k \leq n - 1$ \\
		вектори $  s^0, s^1, \ldots, s^{k-1} $ -  взаємно спряжені \\
		градієнти  $  \phi'(x^0), \phi'(x^1), \ldots,  \phi'(x^{k-1}) $ - взаємно ортогональні
		\item За теоремою \ref{t2} $(\phi'(x^k), \phi'(x^{k-1})) = 0 $ \\
		при $ i \leq k - 2 $, використовуючи (2.1.6), (2.0.2) та індукцію маємо:
		$$ ( \phi'(x^k),  \phi'(x^i)) = (\phi'(x^{k-1}),  \phi'(x^i)) + \beta_{k-1}(As^{k-1},  \phi'(x^i)) = $$
		$$ = \beta_{k-1}(As^{k-1}, - s^k + \gamma_1^{k-1}s^{k-1} + \gamma_2^{k-2}s^{k-2} + \gamma_3^{k-3}s^{k-3}) = 0 $$ 
		
		Взаємна ортогональність векторів  $  \phi'(x^0), \phi'(x^1), \ldots,  \phi'(x^{k-1}) $ доведена. \\
		Вектор $s^k \neq 0 $, інакше вектори  $  \phi'(x^0), \ldots,  \phi'(x^k) $  були б лінійнозалежними ( враховуючи (2.0.1) ), а це суперечить їх взаємній ортогональності.
		
		Доведемо, що вектори $  s^0, \ldots, s^k $  взаємно спряжені.
		
		За (2.1.2) $ (s^k, As^{k-1}) = 0 $, враховуючи (2.1.7) маємо:		
		$$ \beta_i = - \dfrac{(\phi'(x^i), s^i)}{(As^i, s^i)} = -  \dfrac{(\phi'(x^i), -\phi'(x^i) - \gamma_1^{k-1}\phi'(x^i) - \ldots )}{(As^i, s^i)} = $$
		$$ = \dfrac{(\phi'(x^i),\phi'(x^i))}{(As^i, s^i)} $$ 
		з цього випливає, що $ \beta_i \neq 0, \; i \leq k  $, тоді з (2.1.6) отримаємо:
		$$ As^i = \dfrac{(\phi'(x^{i+1} - \phi'(x^i), s^i)}{\beta_i} \eqno(2.1.10) $$
		При  $ i \leq k - 2 $, використовуючи (2.0.1), індукцію і (2.1.10), та доведену взаємну ортогональність градієнтів, отримаємо:
		$$ (s^k, As^i) = (- \phi'(x^k) + \gamma_1^{k-1} s^{k-1} + \gamma_2^{k-2} s^{k-2} + \gamma_3^{k-3} s^{k-3}, As^i) = $$
		$$ = - \left( \phi'(x^k), \dfrac{(\phi'(x^{i+1} - \phi'(x^i), s^i)}{\beta_i} \right)  = 0 $$ 
	\end{enumerate}		
\end{proof}

Отже розглянутий чотирьохкроковий метод (2.0.1), (2.1.2), (2.1.3), (2.1.4) належить до методів спряжених напрямків.

Тепер сформулюємо чотирьохкроковий метод для мінімізації неквадратичних функцій. \\
Для цього перетворимо формули  (2.1.2), (2.1.3), (2.1.4) так, щоб до них не входила матриця $A$.
$$ \gamma_1^{k-1} = \dfrac{(\phi'(x^k), As^{k-1})}{(s^{k-1}, As^{k-1})} = \dfrac{(\phi'(x^k), \phi'(x^k) - \phi'(x^{k-1}))}{(s^{k-1},\phi'(x^k) - \phi'(x^{k-1}))} =  $$
$$ = \dfrac{(\phi'(x^k), \phi'(x^k) - \phi'(x^{k-1}))}{(-\phi'(x^{k-1}) - \gamma_2^{k-2}\phi'(x^i) - \ldots,\phi'(x^k) - \phi'(x^{k-1}))} = $$
$$ = \dfrac{(\phi'(x^k), \phi'(x^k) - \phi'(x^{k-1}))}{\lVert \phi'(x^{k-1}) \rVert^2}  \eqno(2.1.11)$$
далі отримаємо:
$$ \gamma_2^{k-2} = \dfrac{(\phi'(x^k), As^{k-2})}{(s^{k-2}, As^{k-2})} =  \dfrac{(\phi'(x^k), \phi'(x^{k-1}) - \phi'(x^{k-2}))}{\lVert \phi'(x^{k-2}) \rVert^2} \eqno(2.1.12) $$
$$ \gamma_3^{k-3} = \dfrac{(\phi'(x^k), As^{k-3})}{(s^{k-3}, As^{k-3})} =  \dfrac{(\phi'(x^k), \phi'(x^{k-2}) - \phi'(x^{k-3}))}{\lVert \phi'(x^{k-3}) \rVert^2} \eqno(2.1.13) $$

Отже, для неквадратичних функцій, чотирьохкроковий метод має вигляд (2.0.1), (2.1.11), (2.1.12), (2.1.13)

\section{Обчислювальний експеримент}

Зробимо аналіз роботи описаного чотирикрокового методу порівнянно із трикроковим методом. Для порівняння використаємо такі показники, як точність отриманного розв'язку та кількість проведених ітерацій. Під кількістю ітерацій будемо розуміти кількість отриманих послідовних наближень до розв'язку задачі. Зауважимо, що кількість обчислень значень функції та її градієнту у чотирикроковому методі не змінюється порівняно із трикроковим.

Для розрахунків використовувалася мова програмування Python. Обчислення проводилися із точністю $\varepsilon = 10^{-6}, 10^{-8}$. Критерій зупинки процесу обчислень полягав у одночасному виконанні трьох умов:
$$
\phi(x^{k-1}) - \phi(x^{k}) < \varepsilon(1 + |\phi(x^{k})|), $$
$$\lVert x^{k-1} - x^{k} \rVert < \sqrt{\varepsilon}(1 + \lVert x^{k} \rVert), $$
$$ \lVert \phi'(x^{k})\rVert \leq \sqrt[3]{\varepsilon}(1 + |\phi(x^{k})|) 
$$

Для більш ефективного порівняння результати обчислень для кожної початкової точки зведемо у таблиці та проілюструємо гравічно. На графіках вісь абсцис показує кількість ітерацій, вісь ординат - логарифм відхилення значення цільової функції у $i$-й точці від оптимального. Пунктирною лінією позначений процес мінімізіції трикроковим методом, суцільною - чотирикроковим.

\begin{example}
	$$\phi(x) = \left(- x_{1} + 1\right)^{2} + 100 \left(- x_{1}^{2} + x_{2}\right)^{2}$$
	Точний розв'язок задачі: x* = [1, 1] f* = 0
\end{example}
\rotatebox{90}{
	\begin{minipage}{1.375\linewidth}
\begin{table}[H]
	\begin{center}
		\begin{tabular}{|c|c|c|c|c|c|c|}
			\hline
			$\varepsilon$ & $x_0$ & $f(x_0)$ & Метод мінімізації & $x^* \text{ - отриманий розв'язок} $  & $f(x^*)$ & Кількість ітерацій \\
			\hline
			\multirow{8}{*}{$10^{-6}$} & \multirow{2}{*}{[-1.2, 1]} & \multirow{2}{*}{24.2} & 4 кроковий & [ 1.00027  1.00053] & 0.0 & 93 \\
			\hhline{~~~----} & & & 3 кроковий & [ 1.00069  1.00137] & 0.0 & 29 \\
			\hhline{~------}			
			& \multirow{2}{*}{[1.0, -1.2]} & \multirow{2}{*}{484.0} & 4 кроковий & [ 0.99995  0.99991] & 0.0 & 81 \\
			\hhline{~~~----} & & & 3 кроковий & [ 0.99994  0.99988] & 0.0 & 21 \\
			\hhline{~------}
			& \multirow{2}{*}{[0.0, 0.0]} & \multirow{2}{*}{1.0} & 4 кроковий & [ 1.00015  1.00029] & 0.0 & 42 \\
			\hhline{~~~----} & & & 3 кроковий & [ 1.00014  1.00029] & 0.0 & 24 \\
			\hhline{~------}	
			& \multirow{2}{*}{[-1.0, -1.0]} & \multirow{2}{*}{404.0} & 4 кроковий & [ 0.99998  0.99995] & 0.0 & 59 \\
			\hhline{~~~----} & & & 3 кроковий & [ 0.99992  0.99985] & 0.0 & 25 \\
			\hline	
		\end{tabular}
	\end{center}
\caption{}
\end{table}
\end{minipage}} 

\newpage
\begin{figure}[H]
	\begin{minipage}[h]{0.47\linewidth}
		\center{\includegraphics[width=1\linewidth]{image}} a) \\
	\end{minipage}
	\hfill
	\begin{minipage}[h]{0.47\linewidth}
		\center{\includegraphics[width=1\linewidth]{image}} \\b)
	\end{minipage}
	\vfill
	\begin{minipage}[h]{0.47\linewidth}
		\center{\includegraphics[width=1\linewidth]{image}} c) \\
	\end{minipage}
	\hfill
	\begin{minipage}[h]{0.47\linewidth}
		\center{\includegraphics[width=1\linewidth]{image}} d) \\
	\end{minipage}
	\caption{Correlation signal peaks: a) numerical experiment, b)
		registered correlation signals, c) intensity distribution of correlation
		signals in numerical experiment, d) correlation signals intensity
		distribution for DCRAW processed data.}
	\label{ris:experimentalcorrelationsignals}
\end{figure}

\newpage
\rotatebox{90}{
	\begin{minipage}{1.375\linewidth}
		\begin{table}[H]
			\begin{center}
				\begin{tabular}{|c|c|c|c|c|c|c|}
					\hline
					$\varepsilon$ & $x_0$ & $f(x_0)$ & Метод мінімізації & $x^* \text{ - отриманий розв'язок} $  & $f(x^*)$ & Кількість ітерацій \\
					\hline
					\multirow{8}{*}{$10^{-8}$}	& \multirow{2}{*}{[-1.2, 1]} & \multirow{2}{*}{24.2} & 4 кроковий & [ 0.99999  0.99999] & 0.0 & 98 \\
					\hhline{~~~----} & & & 3 кроковий & [ 0.99997  0.99994] & 0.0 & 35 \\
					\hhline{~------}
					& \multirow{2}{*}{[1.0, -1.2]} & \multirow{2}{*}{484.0} & 4 кроковий & [ 1.       0.99999] & 0.0 & 32 \\
					\hhline{~~~----} & & & 3 кроковий & [ 1.  1.] & 0.0 & 25 \\
					\hhline{~------}
					& \multirow{2}{*}{[0.0, 0.0]} & \multirow{2}{*}{1.0} & 4 кроковий & [ 0.99993  0.99985] & 0.0 & 24 \\
					\hhline{~~~----} & & & 3 кроковий & [ 1.  1.] & 0.0 & 19 \\
					\hhline{~------}
					& \multirow{2}{*}{[-1.0, -1.0]} & \multirow{2}{*}{404.0} & 4 кроковий & [ 0.99998  0.99996] & 0.0 & 20 \\
					\hhline{~~~----} & & & 3 кроковий & [ 1.  1.] & 0.0 & 16 \\
					\hline	
				\end{tabular}
			\end{center}
			\caption{}
		\end{table}
\end{minipage}} 

\newpage
\begin{example}
	$$\phi(x) = \left(x_{1} + 10 x_{2}\right)^{2} + 10 \left(x_{1} - x_{4}\right)^{4} + \left(x_{2} - 2 x_{3}\right)^{4} + 5 \left(x_{3} - x_{4}\right)^{2}$$
	Точний розв'язок задачі: x* = [0, 0, 0, 0] f* = 0
\end{example}
\rotatebox{90}{
	\begin{minipage}{1.55\linewidth}
		\begin{table}[H]
			\begin{center}
				\begin{tabular}{|c|c|c|c|c|c|c|}
					\hline
					$\varepsilon$ & $x_0$ & $f(x_0)$ & Метод мінімізації & $x^* \text{ - отриманий розв'язок} $  & $f(x^*)$ & Кількість ітерацій \\
					\hline
				    \multirow{8}{*}{$10^{-6}$} & \multirow{2}{*}{[3, -1, 0, 1]} & \multirow{2}{*}{215} & 4 кроковий & [ 0.04999 -0.00496  0.03379  0.03432] & 3e-05 & 36 \\
				    \hhline{~~~----} & & & 3 кроковий & [ 0.0008  -0.00008 -0.00348 -0.00348] & 0.0 & 36 \\
				    \hhline{~------}
				    & \multirow{2}{*}{[1, 1, 1, 1]} & \multirow{2}{*}{122} & 4 кроковий & [-0.06942  0.00694 -0.04062 -0.04114] & 7e-05 & 43 \\
				    \hhline{~~~----} & & & 3 кроковий & [-0.04808  0.00482 -0.02532 -0.02525] & 1e-05 & 21 \\
				    \hhline{~------}
				    & \multirow{2}{*}{[-1.0, 1.0, -1.0, 1.0]} & \multirow{2}{*}{342.0} & 4 кроковий & [-0.08631  0.00862 -0.04842 -0.04857] & 0.00014 & 38 \\
				    \hhline{~~~----} & & & 3 кроковий & [ 0.00402 -0.00041 -0.01603 -0.01611] & 0.0 & 29 \\
				    \hhline{~------}
				    & \multirow{2}{*}{[0.0, 2.0, -1.0, 1.0]} & \multirow{2}{*}{686.0} & 4 кроковий & [ 0.00241 -0.00024  0.00156  0.00154] & 0.0 & 44 \\
				    \hhline{~~~----} & & & 3 кроковий & [-0.03611  0.00363 -0.00961 -0.00963] & 1e-05 & 27 \\
					\hline	
				\end{tabular}
			\end{center}
			\caption{}
		\end{table}
\end{minipage}} 

\newpage
\rotatebox{90}{
	\begin{minipage}{1.55\linewidth}
		\begin{table}[H]
			\begin{center}
				\begin{tabular}{|c|c|c|c|c|c|c|}
					\hline
					$\varepsilon$ & $x_0$ & $f(x_0)$ & Метод мінімізації & $x^* \text{ - отриманий розв'язок} $  & $f(x^*)$ & Кількість ітерацій \\
					\hline
					\multirow{8}{*}{$10^{-8}$} & \multirow{2}{*}{[3, -1, 0, 1]} & \multirow{2}{*}{215} & 4 кроковий & [-0.02558  0.00256 -0.01165 -0.01175] & 0.0 & 56 \\
					\hhline{~~~----} & & & 3 кроковий & [ 0.00019 -0.00002 -0.00241 -0.00241] & 0.0 & 34 \\
					\hhline{~------}
					& \multirow{2}{*}{[1, 1, 1, 1]} & \multirow{2}{*}{122} & 4 кроковий & [ 0.00081 -0.00008 -0.00795 -0.00794] & 0.0 & 106 \\
					\hhline{~~~----} & & & 3 кроковий & [-0.02441  0.00244 -0.01503 -0.01505] & 0.0 & 37 \\
					\hhline{~------}
					& \multirow{2}{*}{[-1.0, 1.0, -1.0, 1.0]} & \multirow{2}{*}{342.0} & 4 кроковий & [ 0.00905 -0.0009  -0.00682 -0.00667] & 0.0 & 83 \\
					\hhline{~~~----} & & & 3 кроковий & [-0.00833  0.00083 -0.00384 -0.00385] & 0.0 & 44 \\
					\hhline{~------}
					& \multirow{2}{*}{[0.0, 2.0, -1.0, 1.0]} & \multirow{2}{*}{686.0} & 4 кроковий & [ 0.00071 -0.00007  0.00963  0.00962] & 0.0 & 40 \\
					\hhline{~~~----} & & & 3 кроковий & [ 0.00041 -0.00004 -0.00281 -0.00281] & 0.0 & 62 \\
					\hline	
				\end{tabular}
			\end{center}
			\caption{}
		\end{table}
\end{minipage}} 

\newpage
\begin{example}
	$$\phi(x) = \left(x_{1} + x_{2}^{2} - 7\right)^{2} + \left(x_{1}^{2} + x_{2} - 11\right)^{2}$$
	Точний розв'язок задачі:  x* = [3, 2] f* = 0
\end{example}
\rotatebox{90}{
	\begin{minipage}{1.375\linewidth}
		\begin{table}[H]
			\begin{center}
				\begin{tabular}{|c|c|c|c|c|c|c|}
					\hline
					$\varepsilon$ & $x_0$ & $f(x_0)$ & Метод мінімізації & $x^* \text{ - отриманий розв'язок} $  & $f(x^*)$ & Кількість ітерацій \\
					\hline
					\multirow{8}{*}{$10^{-6}$} & \multirow{2}{*}{[1, 1]} & \multirow{2}{*}{106} & 4 кроковий & [ 3.  2.] & 0.0 & 11 \\
					\hhline{~~~----} & & & 3 кроковий & [ 3.  2.] & 0.0 & 9 \\
					\hhline{~------}
					& \multirow{2}{*}{[1.0, 4.0]} & \multirow{2}{*}{136.0} & 4 кроковий & [ 3.  2.] & 0.0 & 17 \\
					\hhline{~~~----} & & & 3 кроковий & [ 3.00011  2.00001] & 0.0 & 9 \\
					\hhline{~------}
					& \multirow{2}{*}{[0.0, 0.0]} & \multirow{2}{*}{170.0} & 4 кроковий & [ 2.99991  1.99991] & 0.0 & 30 \\
					\hhline{~~~----} & & & 3 кроковий & [ 2.99999  2.00016] & 0.0 & 19 \\
					\hhline{~------}
					& \multirow{2}{*}{[2.5, 2.5]} & \multirow{2}{*}{8.125} & 4 кроковий & [ 3.  2.] & 0.0 & 13 \\
					\hhline{~~~----} & & & 3 кроковий & [ 2.99998  2.00004] & 0.0 & 7 \\
					\hline	
				\end{tabular}
			\end{center}
			\caption{}
		\end{table}
\end{minipage}} 

\newpage
\rotatebox{90}{
	\begin{minipage}{1.375\linewidth}
		\begin{table}[H]
			\begin{center}
				\begin{tabular}{|c|c|c|c|c|c|c|}
					\hline
					$\varepsilon$ & $x_0$ & $f(x_0)$ & Метод мінімізації & $x^* \text{ - отриманий розв'язок} $  & $f(x^*)$ & Кількість ітерацій \\
					\hline
					\multirow{8}{*}{$10^{-8}$} & \multirow{2}{*}{[1, 1]} & \multirow{2}{*}{106} & 4 кроковий & [ 3.  2.] & 0.0 & 12 \\
					\hhline{~~~----} & & & 3 кроковий & [ 3.  2.] & 0.0 & 10 \\
					\hhline{~------}
					& \multirow{2}{*}{[1.0, 4.0]} & \multirow{2}{*}{136.0} & 4 кроковий & [ 3.  2.] & 0.0 & 18 \\
					\hhline{~~~----} & & & 3 кроковий & [ 2.99998  2.00003] & 0.0 & 12 \\
					\hhline{~------}
					& \multirow{2}{*}{[0.0, 0.0]} & \multirow{2}{*}{170.0} & 4 кроковий & [ 2.99998  2.     ] & 0.0 & 35 \\
					\hhline{~~~----} & & & 3 кроковий & [ 2.99999  1.99998] & 0.0 & 23 \\
					\hhline{~------}
					& \multirow{2}{*}{[2.5, 2.5]} & \multirow{2}{*}{8.125} & 4 кроковий & [ 3.  2.] & 0.0 & 14 \\
					\hhline{~~~----} & & & 3 кроковий & [ 3.00001  2.00001] & 0.0 & 9 \\
					\hline	
				\end{tabular}
			\end{center}
			\caption{}
		\end{table}
\end{minipage}} 

\newpage
\begin{example}
	$$\phi(x) = \left(x_{1}^{2} + 12 x_{2} - 1\right)^{2} + \left(49 x_{1}^{2} + 84 x_{1} + 49 x_{2}^{2} + 2324 x_{2} - 681\right)^{2} $$
	Точний розв'язок задачі: x* = [0.28581, 0.27936] f* = 5.9225
\end{example}
\rotatebox{90}{
	\begin{minipage}{1.375\linewidth}
		\begin{table}[H]
			\begin{center}
				\begin{tabular}{|c|c|c|c|c|c|c|}
					\hline
					$\varepsilon$ & $x_0$ & $f(x_0)$ & Метод мінімізації & $x^* \text{ - отриманий розв'язок} $  & $f(x^*)$ & Кількість ітерацій \\
					\hline
					\multirow{8}{*}{$10^{-6}$} & \multirow{2}{*}{[1, 1]} & \multirow{2}{*}{3330769} & 4 кроковий & [ 0.28582  0.27933] & 5.92256 & 88 \\
					\hhline{~~~----} & & & 3 кроковий & [ 0.28582  0.27933] & 5.92256 & 26 \\
					\hhline{~------}
					& \multirow{2}{*}{[0.0, 0.0]} & \multirow{2}{*}{463762.0} & 4 кроковий & [ 0.28583  0.27933] & 5.92256 & 70 \\
					\hhline{~~~----} & & & 3 кроковий & [ 0.2858   0.27933] & 5.92256 & 22 \\
					\hhline{~------}
					& \multirow{2}{*}{[-5.0, -7.0]} & \multirow{2}{*}{188873649.0} & 4 кроковий & [ 0.28581  0.27933] & 5.92256 & 91 \\
					\hhline{~~~----} & & & 3 кроковий & [ 0.28581  0.27933] & 5.92256 & 69 \\
					\hhline{~------}
					& \multirow{2}{*}{[0.2, 0.3]} & \multirow{2}{*}{1556.9665} & 4 кроковий & [ 0.28583  0.27933] & 5.92256 & 41 \\
					\hhline{~~~----} & & & 3 кроковий & [ 0.28587  0.27932] & 5.92256 & 21 \\
					\hline	
				\end{tabular}
			\end{center}
			\caption{}
		\end{table}
\end{minipage}} 

\newpage
\rotatebox{90}{
	\begin{minipage}{1.375\linewidth}
		\begin{table}[H]
			\begin{center}
				\begin{tabular}{|c|c|c|c|c|c|c|}
					\hline
					$\varepsilon$ & $x_0$ & $f(x_0)$ & Метод мінімізації & $x^* \text{ - отриманий розв'язок} $  & $f(x^*)$ & Кількість ітерацій \\
					\hline
					\multirow{8}{*}{$10^{-8}$} & \multirow{2}{*}{[1, 1]} & \multirow{2}{*}{3330769} & 4 кроковий & [ 0.28581  0.27933] & 5.92256 & 91 \\
					\hhline{~~~----} & & & 3 кроковий & [ 0.28582  0.27933] & 5.92256 & 27 \\
					\hhline{~------}
					& \multirow{2}{*}{[0.0, 0.0]} & \multirow{2}{*}{463762.0} & 4 кроковий & [ 0.28581  0.27933] & 5.92256 & 73 \\
					\hhline{~~~----} & & & 3 кроковий & [ 0.28582  0.27933] & 5.92256 & 25 \\
					\hhline{~------}
					& \multirow{2}{*}{[-5.0, -7.0]} & \multirow{2}{*}{188873649.0} & 4 кроковий & [ 0.28581  0.27933] & 5.92256 & 152 \\
					\hhline{~~~----} & & & 3 кроковий & [ 0.28581  0.27933] & 5.92256 & 64 \\
					\hhline{~------}
					& \multirow{2}{*}{[0.2, 0.3]} & \multirow{2}{*}{1556.9665} & 4 кроковий & [ 0.28581  0.27933] & 5.92256 & 15 \\
					\hhline{~~~----} & & & 3 кроковий & [ 0.28581  0.27933] & 5.92256 & 12 \\
					\hline	
				\end{tabular}
			\end{center}
			\caption{}
		\end{table}
\end{minipage}} 

\newpage
\begin{example}
	$$\phi(x) = \left(- x_{1} + 1\right)^{2} + \left(- x_{2} + 1\right)^{2} + 100 \left(x_{3} - \left(\frac{x_{1}}{2} + \frac{x_{2}}{2}\right)^{2}\right)^{2}$$
	Точний розв'язок задачі: x* = [1, 1, 1] f* = 0
\end{example}
\rotatebox{90}{
	\begin{minipage}{1.375\linewidth}
		\begin{table}[H]
			\begin{center}
				\begin{tabular}{|c|c|c|c|c|c|c|}
					\hline
					$\varepsilon$ & $x_0$ & $f(x_0)$ & Метод мінімізації & $x^* \text{ - отриманий розв'язок} $  & $f(x^*)$ & Кількість ітерацій \\
					\hline
					\multirow{8}{*}{$10^{-6}$} & \multirow{2}{*}{[-1.2, 2, 0]} & \multirow{2}{*}{8.4} & 4 кроковий & [ 0.99756  0.99819  0.99575] & 1e-05 & 67 \\
					\hhline{~~~----} & & & 3 кроковий & [ 1.00002  0.99978  0.99979] & 0.0 & 31 \\
					\hhline{~------}
					& \multirow{2}{*}{[0.0, 0.0, 0.0]} & \multirow{2}{*}{2.0} & 4 кроковий & [ 1.00009  1.00009  1.00018] & 0.0 & 51 \\
					\hhline{~~~----} & & & 3 кроковий & [ 1.00047  1.00047  1.00091] & 0.0 & 23 \\
					\hhline{~------}
					& \multirow{2}{*}{[0.0, 1.0, -1.2]} & \multirow{2}{*}{211.25} & 4 кроковий & [ 0.99745  1.00094  0.99839] & 1e-05 & 66 \\
					\hhline{~~~----} & & & 3 кроковий & [ 1.00443  0.99998  1.00444] & 2e-05 & 53 \\
					\hhline{~------}
					& \multirow{2}{*}{[2.3, 1.0, -0.3]} & \multirow{2}{*}{915.24062} & 4 кроковий & [ 1.0009   0.9995   1.00042] & 0.0 & 51 \\
					\hhline{~~~----} & & & 3 кроковий & [ 1.00005  0.99997  1.00002] & 0.0 & 24 \\
					\hline	
				\end{tabular}
			\end{center}
			\caption{}
		\end{table}
\end{minipage}} 

\newpage
\rotatebox{90}{
	\begin{minipage}{1.375\linewidth}
		\begin{table}[H]
			\begin{center}
				\begin{tabular}{|c|c|c|c|c|c|c|}
					\hline
					$\varepsilon$ & $x_0$ & $f(x_0)$ & Метод мінімізації & $x^* \text{ - отриманий розв'язок} $  & $f(x^*)$ & Кількість ітерацій \\
					\hline
					\multirow{8}{*}{$10^{-8}$} & \multirow{2}{*}{[-1.2, 2, 0]} & \multirow{2}{*}{8.4} & 4 кроковий & [ 1.00002  0.99975  0.99977] & 0.0 & 96 \\
					\hhline{~~~----} & & & 3 кроковий & [ 1.  1.  1.] & 0.0 & 33 \\
					\hhline{~------}
					& \multirow{2}{*}{[0.0, 0.0, 0.0]} & \multirow{2}{*}{2.0} & 4 кроковий & [ 1.  1.  1.] & 0.0 & 53 \\
					\hhline{~~~----} & & & 3 кроковий & [ 1.0004  1.0004  1.0008] & 0.0 & 26 \\
					\hhline{~------}
					& \multirow{2}{*}{[0.0, 1.0, -1.2]} & \multirow{2}{*}{211.25} & 4 кроковий & [ 1.00002  0.99996  0.99998] & 0.0 & 75 \\
					\hhline{~~~----} & & & 3 кроковий & [ 1.  1.  1.] & 0.0 & 57 \\
					\hhline{~------}
					& \multirow{2}{*}{[2.3, 1.0, -0.3]} & \multirow{2}{*}{915.24062} & 4 кроковий & [ 1.00093  1.00004  1.00097] & 0.0 & 63 \\
					\hhline{~~~----} & & & 3 кроковий & [ 1.00004  0.99996  1.     ] & 0.0 & 25 \\
					\hline	
				\end{tabular}
			\end{center}
			\caption{}
		\end{table}
\end{minipage}} 

\newpage
\begin{example}
	$$\phi(x) = 1352.99688810716 x_{1}^{2} - 70000.0 + \frac{319.284802043423 x_{1}^{2} + 1.67624521072797 x_{2}^{2} + 0.00880028735632184 x_{3}^{2}}{0.00525 x_{4}^{2} + 1} + \frac{416.319733555371 x_{1}^{2} + 1.44879267277269 x_{2}^{2} + 0.00504179850124896 x_{3}^{2}}{0.00348 x_{4}^{2} + 1} + \frac{450.45045045045 x_{1}^{2} + 0.941441441441441 x_{2}^{2} + 0.00196761261261261 x_{3}^{2}}{0.00209 x_{4}^{2} + 1} + \frac{617.283950617284 x_{1}^{2} + 0.993827160493827 x_{2}^{2} + 0.00160006172839506 x_{3}^{2}}{0.00161 x_{4}^{2} + 1} + \frac{608.272506082725 x_{1}^{2} + 0.608272506082725 x_{2}^{2} + 0.000608272506082725 x_{3}^{2}}{0.001 x_{4}^{2} + 1} + \frac{894.454382826476 x_{1}^{2} + 0.382826475849732 x_{2}^{2} + 0.000163849731663685 x_{3}^{2}}{0.000428 x_{4}^{2} + 1}$$
	Точний розв'язок задачі: x* = [0, 0, 0, 1] f* = -70000
\end{example}
\rotatebox{90}{
	\begin{minipage}{1.375\linewidth}
		\begin{table}[H]
			\begin{center}
				\begin{tabular}{|c|c|c|c|c|c|c|}
					\hline
					$\varepsilon$ & $x_0$ & $f(x_0)$ & Метод мінімізації & $x^* \text{ - отриманий розв'язок} $  & $f(x^*)$ & Кількість ітерацій \\
					\hline
					\multirow{8}{*}{$10^{-6}$} & \multirow{2}{*}{[2.7, 90, 1500, 10]} & \multirow{2}{*}{28580.05033} & 4 кроковий & [    0.13676   -15.05046  1495.27438   331.41988] & -69844.78395 & 12 \\
					\hhline{~~~----} & & & 3 кроковий & [   -0.00089    25.86293  1488.12225   473.17638] & -69930.46023 & 26 \\
					\hhline{~------}
					& \multirow{2}{*}{[2.0, 140.0, 1707.0, 31.0]} & \multirow{2}{*}{-8899.54441} & 4 кроковий & [    0.00941   -28.76126  1702.29196   688.04944] & -69957.11694 & 24 \\
					\hhline{~~~----} & & & 3 кроковий & [    0.00421   -35.77071  1702.12319   709.41251] & -69956.78854 & 15 \\
					\hhline{~------}
					& \multirow{2}{*}{[1.0, 1.0, 1.0, 1.0]} & \multirow{2}{*}{-65340.91918} & 4 кроковий & [ 0.       0.00001  0.00561  1.00689] & -70000.0 & 8 \\
					\hhline{~~~----} & & & 3 кроковий & [-0.       0.00001  0.0015   1.00691] & -70000.0 & 7 \\
					\hhline{~------}
					& \multirow{2}{*}{[-1.0, 0.0, 1.0, -1.0]} & \multirow{2}{*}{-65346.95247} & 4 кроковий & [-0.00001  0.       0.00001 -1.00519] & -70000.0 & 4 \\
					\hhline{~~~----} & & & 3 кроковий & [ 0.       0.       0.      -1.00519] & -70000.0 & 4 \\
					\hline	
				\end{tabular}
			\end{center}
			\caption{}
		\end{table}
\end{minipage}} 

\newpage
\rotatebox{90}{
	\begin{minipage}{1.375\linewidth}
		\begin{table}[H]
			\begin{center}
				\begin{tabular}{|c|c|c|c|c|c|c|}
					\hline
					$\varepsilon$ & $x_0$ & $f(x_0)$ & Метод мінімізації & $x^* \text{ - отриманий розв'язок} $  & $f(x^*)$ & Кількість ітерацій \\
					\hline
					\multirow{8}{*}{$10^{-8}$} & \multirow{2}{*}{[2.7, 90, 1500, 10]} & \multirow{2}{*}{28580.05033} & 4 кроковий & [   -0.00743     4.95678  1492.21697   388.69679] & -69910.52408 & 33 \\
					\hhline{~~~----} & & & 3 кроковий & [   -0.00011    -8.10017  1403.31712   696.40143] & -69975.00883 & 86 \\
					\hhline{~------}
					& \multirow{2}{*}{[2.0, 140.0, 1707.0, 31.0]} & \multirow{2}{*}{-8899.54441} & 4 кроковий & [   -0.00397   -32.08949  1702.15048   590.31179] & -69940.00293 & 28 \\
					\hhline{~~~----} & & & 3 кроковий & [   -0.00018    -3.70682  1679.4614   1104.81388] & -69985.98546 & 88 \\
					\hhline{~------}
					& \multirow{2}{*}{[1.0, 1.0, 1.0, 1.0]} & \multirow{2}{*}{-65340.91918} & 4 кроковий & [ 0.       0.00001  0.00561  1.00689] & -70000.0 & 8 \\
					\hhline{~~~----} & & & 3 кроковий & [-0.       0.00001  0.0015   1.00691] & -70000.0 & 7 \\
					\hhline{~------}
					& \multirow{2}{*}{[-1.0, 0.0, 1.0, -1.0]} & \multirow{2}{*}{-65346.95247} & 4 кроковий & [-0.00001  0.       0.00001 -1.00519] & -70000.0 & 4 \\
					\hhline{~~~----} & & & 3 кроковий & [ 0.       0.       0.      -1.00519] & -70000.0 & 4 \\
					\hline	
				\end{tabular}
			\end{center}
			\caption{}
		\end{table}
\end{minipage}} 

\newpage
\begin{example}
	$$\phi(x) = $$
	Точний розв'язок задачі: 
\end{example}
\rotatebox{90}{
	\begin{minipage}{1.375\linewidth}
		\begin{table}[H]
			\begin{center}
				\begin{tabular}{|c|c|c|c|c|c|c|}
					\hline
					$\varepsilon$ & $x_0$ & $f(x_0)$ & Метод мінімізації & $x^* \text{ - отриманий розв'язок} $  & $f(x^*)$ & Кількість ітерацій \\
					\hline
					\multirow{8}{*}{$10^{-6}$}
					
					\hline	
				\end{tabular}
			\end{center}
			\caption{}
		\end{table}
\end{minipage}} 

\newpage
\rotatebox{90}{
	\begin{minipage}{1.375\linewidth}
		\begin{table}[H]
			\begin{center}
				\begin{tabular}{|c|c|c|c|c|c|c|}
					\hline
					$\varepsilon$ & $x_0$ & $f(x_0)$ & Метод мінімізації & $x^* \text{ - отриманий розв'язок} $  & $f(x^*)$ & Кількість ітерацій \\
					\hline
					\multirow{8}{*}{$10^{-8}$}
					
					\hline	
				\end{tabular}
			\end{center}
			\caption{}
		\end{table}
\end{minipage}} 

\newpage
\begin{example}
	$$\phi(x) = $$
	Точний розв'язок задачі: 
\end{example}
\rotatebox{90}{
	\begin{minipage}{1.375\linewidth}
		\begin{table}[H]
			\begin{center}
				\begin{tabular}{|c|c|c|c|c|c|c|}
					\hline
					$\varepsilon$ & $x_0$ & $f(x_0)$ & Метод мінімізації & $x^* \text{ - отриманий розв'язок} $  & $f(x^*)$ & Кількість ітерацій \\
					\hline
					\multirow{8}{*}{$10^{-6}$}
					
					\hline	
				\end{tabular}
			\end{center}
			\caption{}
		\end{table}
\end{minipage}} 

\newpage
\rotatebox{90}{
	\begin{minipage}{1.375\linewidth}
		\begin{table}[H]
			\begin{center}
				\begin{tabular}{|c|c|c|c|c|c|c|}
					\hline
					$\varepsilon$ & $x_0$ & $f(x_0)$ & Метод мінімізації & $x^* \text{ - отриманий розв'язок} $  & $f(x^*)$ & Кількість ітерацій \\
					\hline
					\multirow{8}{*}{$10^{-8}$}
					
					\hline	
				\end{tabular}
			\end{center}
			\caption{}
		\end{table}
\end{minipage}} 

\newpage
\begin{example}
	$$\phi(x) = $$
	Точний розв'язок задачі: 
\end{example}
\rotatebox{90}{
	\begin{minipage}{1.375\linewidth}
		\begin{table}[H]
			\begin{center}
				\begin{tabular}{|c|c|c|c|c|c|c|}
					\hline
					$\varepsilon$ & $x_0$ & $f(x_0)$ & Метод мінімізації & $x^* \text{ - отриманий розв'язок} $  & $f(x^*)$ & Кількість ітерацій \\
					\hline
					\multirow{8}{*}{$10^{-6}$}
					
					\hline	
				\end{tabular}
			\end{center}
			\caption{}
		\end{table}
\end{minipage}} 

\newpage
\rotatebox{90}{
	\begin{minipage}{1.375\linewidth}
		\begin{table}[H]
			\begin{center}
				\begin{tabular}{|c|c|c|c|c|c|c|}
					\hline
					$\varepsilon$ & $x_0$ & $f(x_0)$ & Метод мінімізації & $x^* \text{ - отриманий розв'язок} $  & $f(x^*)$ & Кількість ітерацій \\
					\hline
					\multirow{8}{*}{$10^{-8}$}
					
					\hline	
				\end{tabular}
			\end{center}
			\caption{}
		\end{table}
\end{minipage}} 

\newpage
\begin{example}
	$$\phi(x) = $$
	Точний розв'язок задачі: 
\end{example}
\rotatebox{90}{
	\begin{minipage}{1.375\linewidth}
		\begin{table}[H]
			\begin{center}
				\begin{tabular}{|c|c|c|c|c|c|c|}
					\hline
					$\varepsilon$ & $x_0$ & $f(x_0)$ & Метод мінімізації & $x^* \text{ - отриманий розв'язок} $  & $f(x^*)$ & Кількість ітерацій \\
					\hline
					\multirow{8}{*}{$10^{-6}$}
					
					\hline	
				\end{tabular}
			\end{center}
			\caption{}
		\end{table}
\end{minipage}} 

\newpage
\rotatebox{90}{
	\begin{minipage}{1.375\linewidth}
		\begin{table}[H]
			\begin{center}
				\begin{tabular}{|c|c|c|c|c|c|c|}
					\hline
					$\varepsilon$ & $x_0$ & $f(x_0)$ & Метод мінімізації & $x^* \text{ - отриманий розв'язок} $  & $f(x^*)$ & Кількість ітерацій \\
					\hline
					\multirow{8}{*}{$10^{-8}$}
					
					\hline	
				\end{tabular}
			\end{center}
			\caption{}
		\end{table}
\end{minipage}} 

\newpage
\begin{example}
	$$\phi(x) = $$
	Точний розв'язок задачі: 
\end{example}
\rotatebox{90}{
	\begin{minipage}{1.375\linewidth}
		\begin{table}[H]
			\begin{center}
				\begin{tabular}{|c|c|c|c|c|c|c|}
					\hline
					$\varepsilon$ & $x_0$ & $f(x_0)$ & Метод мінімізації & $x^* \text{ - отриманий розв'язок} $  & $f(x^*)$ & Кількість ітерацій \\
					\hline
					\multirow{8}{*}{$10^{-6}$}
					
					\hline	
				\end{tabular}
			\end{center}
			\caption{}
		\end{table}
\end{minipage}} 

\newpage
\rotatebox{90}{
	\begin{minipage}{1.375\linewidth}
		\begin{table}[H]
			\begin{center}
				\begin{tabular}{|c|c|c|c|c|c|c|}
					\hline
					$\varepsilon$ & $x_0$ & $f(x_0)$ & Метод мінімізації & $x^* \text{ - отриманий розв'язок} $  & $f(x^*)$ & Кількість ітерацій \\
					\hline
					\multirow{8}{*}{$10^{-8}$}
					
					\hline	
				\end{tabular}
			\end{center}
			\caption{}
		\end{table}
\end{minipage}} 

\newpage
\begin{example}
	$$\phi(x) = $$
	Точний розв'язок задачі: 
\end{example}
\rotatebox{90}{
	\begin{minipage}{1.375\linewidth}
		\begin{table}[H]
			\begin{center}
				\begin{tabular}{|c|c|c|c|c|c|c|}
					\hline
					$\varepsilon$ & $x_0$ & $f(x_0)$ & Метод мінімізації & $x^* \text{ - отриманий розв'язок} $  & $f(x^*)$ & Кількість ітерацій \\
					\hline
					\multirow{8}{*}{$10^{-6}$}
					
					\hline	
				\end{tabular}
			\end{center}
			\caption{}
		\end{table}
\end{minipage}} 

\newpage
\rotatebox{90}{
	\begin{minipage}{1.375\linewidth}
		\begin{table}[H]
			\begin{center}
				\begin{tabular}{|c|c|c|c|c|c|c|}
					\hline
					$\varepsilon$ & $x_0$ & $f(x_0)$ & Метод мінімізації & $x^* \text{ - отриманий розв'язок} $  & $f(x^*)$ & Кількість ітерацій \\
					\hline
					\multirow{8}{*}{$10^{-8}$}
					
					\hline	
				\end{tabular}
			\end{center}
			\caption{}
		\end{table}
\end{minipage}} 

\newpage
\begin{example}
	$$\phi(x) = $$
	Точний розв'язок задачі: 
\end{example}
\rotatebox{90}{
	\begin{minipage}{1.375\linewidth}
		\begin{table}[H]
			\begin{center}
				\begin{tabular}{|c|c|c|c|c|c|c|}
					\hline
					$\varepsilon$ & $x_0$ & $f(x_0)$ & Метод мінімізації & $x^* \text{ - отриманий розв'язок} $  & $f(x^*)$ & Кількість ітерацій \\
					\hline
					\multirow{8}{*}{$10^{-6}$}
					
					\hline	
				\end{tabular}
			\end{center}
			\caption{}
		\end{table}
\end{minipage}} 

\newpage
\rotatebox{90}{
	\begin{minipage}{1.375\linewidth}
		\begin{table}[H]
			\begin{center}
				\begin{tabular}{|c|c|c|c|c|c|c|}
					\hline
					$\varepsilon$ & $x_0$ & $f(x_0)$ & Метод мінімізації & $x^* \text{ - отриманий розв'язок} $  & $f(x^*)$ & Кількість ітерацій \\
					\hline
					\multirow{8}{*}{$10^{-8}$}
					
					\hline	
				\end{tabular}
			\end{center}
			\caption{}
		\end{table}
\end{minipage}} 

\newpage
\begin{example}
	$$\phi(x) = $$
	Точний розв'язок задачі: 
\end{example}
\rotatebox{90}{
	\begin{minipage}{1.375\linewidth}
		\begin{table}[H]
			\begin{center}
				\begin{tabular}{|c|c|c|c|c|c|c|}
					\hline
					$\varepsilon$ & $x_0$ & $f(x_0)$ & Метод мінімізації & $x^* \text{ - отриманий розв'язок} $  & $f(x^*)$ & Кількість ітерацій \\
					\hline
					\multirow{8}{*}{$10^{-6}$}
					
					\hline	
				\end{tabular}
			\end{center}
			\caption{}
		\end{table}
\end{minipage}} 

\newpage
\rotatebox{90}{
	\begin{minipage}{1.375\linewidth}
		\begin{table}[H]
			\begin{center}
				\begin{tabular}{|c|c|c|c|c|c|c|}
					\hline
					$\varepsilon$ & $x_0$ & $f(x_0)$ & Метод мінімізації & $x^* \text{ - отриманий розв'язок} $  & $f(x^*)$ & Кількість ітерацій \\
					\hline
					\multirow{8}{*}{$10^{-8}$}
					
					\hline	
				\end{tabular}
			\end{center}
			\caption{}
		\end{table}
\end{minipage}} 

