\chapter{Чотирьохкроковий метод мінімізації функцій багатьох змінних без обмежень}

Розглянемо задачу мінімізації функції багатьох змінних без обмежень
$$ \phi (x) \to min, \; x \in \mathbb{R}^n , \eqno(2.0.1) $$ де  
$$ \phi (x) : \mathbb{R}^n \to \mathbb{R}^1 \text{ - неперервно диференційовна функція } $$
Для її розв'язання розглянемо чотирьохкроковий метод з таким алгоритмом:
$$ x^{k + 1} = x^k + \beta_k s^k, \; k = 0, 1,\ldots $$
$$
S = \quad
\begin{cases}
-\phi' (x^k) & , k = 0 \\
-\phi' (x^k) + \gamma_1^{k-1}s^{k-1} & , k = 1 \\
-\phi' (x^k) + \gamma_1^{k-1}s^{k-1} + \gamma_2^{k-2}s^{k-2} & , k = 2 \\
-\phi' (x^k) + \gamma_1^{k-1}s^{k-1} + \gamma_2^{k-2}s^{k-2} +  \gamma_3^{k-3}s^{k-3} & , k = 3, \ldots
\end{cases}	\eqno(2.0.2)
$$
де $ x^0, x^1, \dotsc , x^k, \dotsc $ - послідовні наближення \\
$ s^0, s^1, \dotsc , s^k, \dotsc $ - напрямки спуску \\
$ \beta_k, \gamma_j^m (j = \overline{1,3}) $ - числові параметри 

Параметр $\beta_k$ будемо визначати з умови:
$$ \beta_k : \min_{\beta\geq0} \phi(x^k + \beta s^k) \eqno(2.0.3)$$
де $  \phi'(x) $ - градієнт функції $ \phi(x) $

\begin{defn}\label{conjugatev}
	Вектори $ s' \text{ і} \; s'' $ називаються спряженими (відносно матриці $A$), якщо вони відмінні від нуля і $(As', s'') = 0$. \\
	Вектори $ s^0, s^1, ..., s^m $ називаються взаємно спряженими (відносно матриці $A$), якщо всі вони відмінні від нуля і $(As', s'') = 0, i \neq j, 0 \leq i,j \geq m$. Матриця $A$ вважається симетричною і додатньо визначеною $(A > 0).$
\end{defn}

\section{Теоретичні положення методу}

Розглянемо деякі властивості методу при умові, що функція $\phi(x)$ є квадратичною.
$$  \phi(x) = \frac{1}{2} (Ax,x) + (b,x) + c \eqno(2.1.1) $$
Побудуємо систему взаємно спряжених напрямків за правилом (2.0.2). 
$$ 0 = (s^k, As^{k-1}) = -(\phi'(x^k), As^{k-1}) + \gamma_1^{k-1}(s^{k-1}, As^{k-1}) + $$
$$ + \gamma_2^{k-2}\underbrace{(s^{k-2}, As^{k-1})}_{0} + \gamma_3^{k-3}\underbrace{(s^{k-3}, As^{k-1})}_{0}  \Longrightarrow$$ 
$$\gamma_1^{k-1} = \dfrac{(\phi'(x^k), As^{k-1})}{(s^{k-1}, As^{k-1})} \eqno(2.1.2)$$
Внаслідок того, що матриця $A$ додатньо визначена, знаменник у (2.1.2) не дорівнює нулеві. 
Аналогічно отримаємо: 
$$ 0 = (s^k, As^{k-2}) = -(\phi'(x^k), As^{k-2}) + \gamma_1^{k-1}\underbrace{(s^{k-1}, As^{k-2})}_{0} + $$
$$ + \gamma_2^{k-2}(s^{k-2}, As^{k-2}) + \gamma_3^{k-3}\underbrace{(s^{k-3}, As^{k-2})}_{0}  \Longrightarrow$$  
$$\gamma_2^{k-2} = \dfrac{(\phi'(x^k), As^{k-2})}{(s^{k-2}, As^{k-2})} \eqno(2.1.3)$$

$$ 0 = (s^k, As^{k-3}) = -(\phi'(x^k), As^{k-3}) + \gamma_1^{k-1}\underbrace{(s^{k-1}, As^{k-3})}_{0} + $$
$$ + \gamma_2^{k-2}\underbrace{(s^{k-2}, As^{k-3})}_{0} + \gamma_3^{k-3}(s^{k-3}, As^{k-3})  \Longrightarrow$$  
$$\gamma_3^{k-3} = \dfrac{(\phi'(x^k), As^{k-3})}{(s^{k-3}, As^{k-3})} \eqno(2.1.4)$$

\newpage 
\begin{thm} \label{t1}
	Для диференційовної функції $\phi(x)$ послідовність $\{x^k\}$, що побудована за (2.0.2), (2.1.2), (2.1.3), (2.1.4), така, що $$ (\phi'(x^{k+1}), s^k) = 0, \, k = 0, 1, \ldots \eqno(2.1.5)$$ 
\end{thm}
\begin{proof}[Доведення:]
	Враховуючи (2.0.2) отримаємо $Ax^{k+1} = Ax^k + \beta_kAs^k$. Так як $\phi'(x) = Ax + b$, то маємо $$ \phi'(x^{k+1}) = \phi'(x) + \beta_k  A  s^k \eqno(2.1.6)$$
	З (2.0.3) отримаємо, що 	
	\begin{align*}
	\dfrac{d}{d \beta} \phi(x^k + \beta s^k)&\Bigr|_{\beta = \beta_k} = 0 & , \; \beta_k& > 0 \\
	\dfrac{d}{d \beta} \phi(x^k + \beta s^k)&\Bigr|_{\beta < 0} \; \; \geq 0 & , \; \beta_k& = 0
	\end{align*}	
	Якщо $\beta_k > 0$, то
	$$ 0 = \dfrac{d}{d \beta} \phi(x^k + \beta s^k) \Bigr|_{\beta = \beta_k} = ( \phi'(x^k + \beta_k s^k), s^k) = (\phi'(x^{k+1}), s^k) $$
	Отже, отримали, що $(\phi'(x^{k+1}), s^k)  = 0, \; k = 0, 1, \ldots$
	
	Застосуємо метод індукції для доведення того, що співвідношення (2.1.5) справедливе і при $\beta_k = 0$:
	
	\begin{enumerate}
		{% 
			\renewcommand{\baselinestretch}{1.7}
			\selectfont
			\item $ 0 \leq \dfrac{d}{d \beta} \phi(x^0 + \beta s^0) \Bigr|_{\beta = 0} = ( \phi'(x^1), s^0) = (\phi'(x^0), -\phi'(x^0)) = $
			
			$ =  -\lVert \phi'(x^0) \rVert^2 \Rightarrow (\phi'(x^1), s^0)  = 0$
			\item припустимо, що $(\phi'(x^k), s^{k-1})  = 0$
			\item доведемо, що $(\phi'(x^{k+1}), s^k)  = 0, \beta_k = 0$
			
		}%

		Так, як $x^{k+1} = x^k$, то враховуючи (2.1.6) маємо:
		
		$$ 0 \leq \dfrac{d}{d \beta} \phi(x^k + \beta s^k) \Bigr|_{\beta = 0} =  ( \phi'(x^{k+1}), s^k) = ( \phi'(x^k), s^k) = $$
		$$ = ( \phi'(x^k), - \phi'(x^k) + \gamma_1^{k-1}s^{k-1} + \gamma_2^{k-2}s^{k-2} + \gamma_3^{k-3}s^{k-3}) = $$
		$$ = -( \phi'(x^k), \phi'(x^k)) + \gamma_1^{k-1}( \phi'(x^k),s^{k-1}) + $$
		$$ + \gamma_2^{k-2}( \phi'(x^k),s^{k-2}) + \gamma_3^{k-3}( \phi'(x^k),s^{k-3}) $$
		Враховуючи (2.1.6) і припущення індукції маємо:		
		$$( \phi'(x^k),s^{k-2}) = (\phi'(x^{k-1}) + \beta_{k-1}As^{k-1}, s^{k-2}) = (\phi'(x^{k-1}), s^{k-2}) + $$
		$$ + \beta_{k-1}(As^{k-1}, s^{k-2}) = (\phi'(x^{k-1}), s^{k-2}) = 0 $$
		$$( \phi'(x^k),s^{k-3}) = (\phi'(x^{k-1}) + \beta_{k-1}As^{k-1}, s^{k-3}) = (\phi'(x^{k-1}), s^{k-3}) + $$
		$$ + \beta_{k-1}(As^{k-1}, s^{k-3}) = (\phi'(x^{k-1}), s^{k-3}) = (\phi'(x^{k-2}) + \beta_{k-2}As^{k-2}, s^{k-3})  = $$
		$$ = (\phi'(x^{k-2}), s^{k-3}) + \beta_{k-2}(As^{k-2}, s^{k-3}) = (\phi'(x^{k-2}), s^{k-3}) = 0 $$		
		Отже, отримали:
		$$ 0 \leq (\phi'(x^{k+1}), s^k) = -\lVert \phi'(x^k) \rVert^2 \leq 0 $$
		Таким чином довели, що $ (\phi'(x^{k+1}), s^k) = 0$
	\end{enumerate}		
\end{proof}

\begin{thm} \label{t2}
	Вектори $\phi'(x^k) \; \text{i} \; \phi'(x^{k+1}) $ ортогональні, $ k = 0, 1, \ldots $ 
\end{thm}
\begin{proof}[Доведення:]
	Відомо, що квадратична функція (2.1.1) досягає мінімального значення при
	$$ \beta_k = -\dfrac{(\phi'(x^k), s^k)}{(As^k, s^k)} \eqno(2.1.7) $$
	
	Тоді враховуючи (2.1.6) та (2.1.7), отримаємо:	
	$$
	(\phi'(x^{k+1}), \phi'(x^k)) = (\phi'(x^k) + \beta_kAs^k,  \phi'(x^k)) = 
	$$
	$$ 
	 = (\phi'(x^k) - \dfrac{(\phi'(x^k), s^k)}{(As^k, s^k)}As^k,  \phi'(x^k)) = 
	$$
	$$	
	= (\phi'(x^k), \phi'(x^k)) - \dfrac{(\phi'(x^k), s^k)}{(As^k, s^k)}(As^k,  \phi'(x^k))
	$$		
	
	Розглянємо $ (\phi'(x^k), s^k) $.
	
	$$ (\phi'(x^k), s^k) = - (\phi'(x^k), \phi'(x^k)) + \gamma_1^{k-1}\underbrace{(\phi'(x^k), s^{k-1})}_{0} + $$
	$$ + \gamma_2^{k-2}\underbrace{(\phi'(x^k), s^{k-2})}_{0} + \gamma_3^{k-3}\underbrace{(\phi'(x^k), s^{k-3})}_{0} = 
	$$
	$$ = - (\phi'(x^k), \phi'(x^k)) \eqno(2.1.8) $$
	
	Далі, $$ (As^k, s^k) = (As^k, -\phi'(x^k) + \gamma_1^{k-1}s^{k-1} + \gamma_2^{k-2}s^{k-2} + \gamma_3^{k-3}s^{k-3} ) = $$
	$$ = - (As^k, \phi'(x^k)) + \gamma_1^{k-1}(As^k, s^{k-1}) + \gamma_2^{k-2}(As^k, s^{k-2}) + \gamma_3^{k-3}(As^k, s^{k-3}) = $$
	$$ =  - (As^k, \phi'(x^k)) \eqno(2.1.9) $$
	З урахуванням (2.1.8) і (2.1.9) отримаємо:	
	$$  (\phi'(x^{k+1}), \phi'(x^k)) =  (\phi'(x^k), \phi'(x^k)) - \dfrac{-(\phi'(x^k),\phi'(x^k))}{-(As^k,\phi'(x^k))}(As^k, \phi'(x^k)) = 0$$ 
\end{proof}

\begin{thm} \label{t3}
	$ \text{Нехай } x^0 \in \mathbb{R}^n , \text{ точки } x^1, x^2, \ldots , x^{n-1} \text{ і вектори } s^0, s^1, \ldots, s^{n-1} $ отримані за формулами  (2.0.2), (2.1.2), (2.1.3), (2.1.4) і 
	$ \phi'(x^k) \neq 0 \\ (k = \overline{0,n-1}) $, тоді вектори $  s^0, s^1, \ldots, s^{n-1} $ взаємно спряжені, а градієнти \\ $  \phi'(x^0), \phi'(x^1), \ldots,  \phi'(x^{n-1}) $ взаємно ортогональні.
\end{thm}
\begin{proof}[Доведення:]
	Теорему будемо доводити методом математичної індукції.
	\begin{enumerate}
		\item $  \phi'(x^0) \; і \; \phi'(x^1) $ - ортогональні внаслідок теореми \ref{t2} \\
		$s^0 \neq 0 $ - за умовою теореми \\
		$s^1 \neq 0 $, так як $s^1 = - \phi'(x^1) - \gamma_0\phi'(x^0) = 0 $, а це неможливо, враховуючи ортогональність  $  \phi'(x^0) \; і \; \phi'(x^1) $ \\
		спряженість $s^0 \; і \; s^1 $ отримаємо з (2.1.2), (2.1.3), (2.1.4) 
		\item Припустимо, що $k \leq n - 1$ \\
		вектори $  s^0, s^1, \ldots, s^{k-1} $ -  взаємно спряжені \\
		градієнти  $  \phi'(x^0), \phi'(x^1), \ldots,  \phi'(x^{k-1}) $ - взаємно ортогональні
		\item За теоремою \ref{t2} $(\phi'(x^k), \phi'(x^{k-1})) = 0 $ \\
		при $ i \leq k - 2 $, використовуючи (2.1.6), (2.0.2) та індукцію маємо:
		$$ ( \phi'(x^k),  \phi'(x^i)) = (\phi'(x^{k-1}),  \phi'(x^i)) + \beta_{k-1}(As^{k-1},  \phi'(x^i)) = $$
		$$ = \beta_{k-1}(As^{k-1}, - s^k + \gamma_1^{k-1}s^{k-1} + \gamma_2^{k-2}s^{k-2} + \gamma_3^{k-3}s^{k-3}) = 0 $$ 
		
		Взаємна ортогональність векторів  $  \phi'(x^0), \phi'(x^1), \ldots,  \phi'(x^{k-1}) $ доведена. \\
		Вектор $s^k \neq 0 $, інакше вектори  $  \phi'(x^0), \ldots,  \phi'(x^k) $  були б лінійнозалежними ( враховуючи (2.0.2) ), а це суперечить їх взаємній ортогональності.
		
		Доведемо, що вектори $  s^0, \ldots, s^k $  взаємно спряжені.
		
		За (2.1.2) $ (s^k, As^{k-1}) = 0 $, враховуючи (2.1.7) маємо:		
		$$ \beta_i = - \dfrac{(\phi'(x^i), s^i)}{(As^i, s^i)} = -  \dfrac{(\phi'(x^i), -\phi'(x^i) - \gamma_1^{k-1}\phi'(x^i) - \ldots )}{(As^i, s^i)} = $$
		$$ = \dfrac{(\phi'(x^i),\phi'(x^i))}{(As^i, s^i)} $$ 
		з цього випливає, що $ \beta_i \neq 0, \; i \leq k  $, тоді з (2.1.6) отримаємо:
		$$ As^i = \dfrac{(\phi'(x^{i+1} - \phi'(x^i), s^i)}{\beta_i} \eqno(2.1.10) $$
		При  $ i \leq k - 2 $, використовуючи (2.0.2), індукцію і (2.1.10), та доведену взаємну ортогональність градієнтів, отримаємо:
		$$ (s^k, As^i) = (- \phi'(x^k) + \gamma_1^{k-1} s^{k-1} + \gamma_2^{k-2} s^{k-2} + \gamma_3^{k-3} s^{k-3}, As^i) = $$
		$$ = - \left( \phi'(x^k), \dfrac{(\phi'(x^{i+1} - \phi'(x^i), s^i)}{\beta_i} \right)  = 0 $$ 
	\end{enumerate}		
\end{proof}

Отже розглянутий чотирьохкроковий метод (2.0.2), (2.1.2), (2.1.3), (2.1.4) належить до методів спряжених напрямків.

Тепер сформулюємо чотирьохкроковий метод для мінімізації неквадратичних функцій. \\
Для цього перетворимо формули  (2.1.2), (2.1.3), (2.1.4) так, щоб до них не входила матриця $A$.
$$ \gamma_1^{k-1} = \dfrac{(\phi'(x^k), As^{k-1})}{(s^{k-1}, As^{k-1})} = \dfrac{(\phi'(x^k), \phi'(x^k) - \phi'(x^{k-1}))}{(s^{k-1},\phi'(x^k) - \phi'(x^{k-1}))} =  $$
$$ = \dfrac{(\phi'(x^k), \phi'(x^k) - \phi'(x^{k-1}))}{(-\phi'(x^{k-1}) - \gamma_2^{k-2}\phi'(x^i) - \ldots,\phi'(x^k) - \phi'(x^{k-1}))} = $$
$$ = \dfrac{(\phi'(x^k), \phi'(x^k) - \phi'(x^{k-1}))}{\lVert \phi'(x^{k-1}) \rVert^2}  \eqno(2.1.11)$$
далі отримаємо:
$$ \gamma_2^{k-2} = \dfrac{(\phi'(x^k), As^{k-2})}{(s^{k-2}, As^{k-2})} =  \dfrac{(\phi'(x^k), \phi'(x^{k-1}) - \phi'(x^{k-2}))}{\lVert \phi'(x^{k-2}) \rVert^2} \eqno(2.1.12) $$
$$ \gamma_3^{k-3} = \dfrac{(\phi'(x^k), As^{k-3})}{(s^{k-3}, As^{k-3})} =  \dfrac{(\phi'(x^k), \phi'(x^{k-2}) - \phi'(x^{k-3}))}{\lVert \phi'(x^{k-3}) \rVert^2} \eqno(2.1.13) $$

Отже, для неквадратичних функцій, чотирьохкроковий метод має вигляд (2.0.2), (2.1.11), (2.1.12), (2.1.13)

\section{Тестові задачі}

